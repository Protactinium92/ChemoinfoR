% Options for packages loaded elsewhere
\PassOptionsToPackage{unicode}{hyperref}
\PassOptionsToPackage{hyphens}{url}
%
\documentclass[
]{article}
\usepackage{amsmath,amssymb}
\usepackage{iftex}
\ifPDFTeX
  \usepackage[T1]{fontenc}
  \usepackage[utf8]{inputenc}
  \usepackage{textcomp} % provide euro and other symbols
\else % if luatex or xetex
  \usepackage{unicode-math} % this also loads fontspec
  \defaultfontfeatures{Scale=MatchLowercase}
  \defaultfontfeatures[\rmfamily]{Ligatures=TeX,Scale=1}
\fi
\usepackage{lmodern}
\ifPDFTeX\else
  % xetex/luatex font selection
\fi
% Use upquote if available, for straight quotes in verbatim environments
\IfFileExists{upquote.sty}{\usepackage{upquote}}{}
\IfFileExists{microtype.sty}{% use microtype if available
  \usepackage[]{microtype}
  \UseMicrotypeSet[protrusion]{basicmath} % disable protrusion for tt fonts
}{}
\makeatletter
\@ifundefined{KOMAClassName}{% if non-KOMA class
  \IfFileExists{parskip.sty}{%
    \usepackage{parskip}
  }{% else
    \setlength{\parindent}{0pt}
    \setlength{\parskip}{6pt plus 2pt minus 1pt}}
}{% if KOMA class
  \KOMAoptions{parskip=half}}
\makeatother
\usepackage{xcolor}
\usepackage[margin=1in]{geometry}
\usepackage{color}
\usepackage{fancyvrb}
\newcommand{\VerbBar}{|}
\newcommand{\VERB}{\Verb[commandchars=\\\{\}]}
\DefineVerbatimEnvironment{Highlighting}{Verbatim}{commandchars=\\\{\}}
% Add ',fontsize=\small' for more characters per line
\usepackage{framed}
\definecolor{shadecolor}{RGB}{248,248,248}
\newenvironment{Shaded}{\begin{snugshade}}{\end{snugshade}}
\newcommand{\AlertTok}[1]{\textcolor[rgb]{0.94,0.16,0.16}{#1}}
\newcommand{\AnnotationTok}[1]{\textcolor[rgb]{0.56,0.35,0.01}{\textbf{\textit{#1}}}}
\newcommand{\AttributeTok}[1]{\textcolor[rgb]{0.13,0.29,0.53}{#1}}
\newcommand{\BaseNTok}[1]{\textcolor[rgb]{0.00,0.00,0.81}{#1}}
\newcommand{\BuiltInTok}[1]{#1}
\newcommand{\CharTok}[1]{\textcolor[rgb]{0.31,0.60,0.02}{#1}}
\newcommand{\CommentTok}[1]{\textcolor[rgb]{0.56,0.35,0.01}{\textit{#1}}}
\newcommand{\CommentVarTok}[1]{\textcolor[rgb]{0.56,0.35,0.01}{\textbf{\textit{#1}}}}
\newcommand{\ConstantTok}[1]{\textcolor[rgb]{0.56,0.35,0.01}{#1}}
\newcommand{\ControlFlowTok}[1]{\textcolor[rgb]{0.13,0.29,0.53}{\textbf{#1}}}
\newcommand{\DataTypeTok}[1]{\textcolor[rgb]{0.13,0.29,0.53}{#1}}
\newcommand{\DecValTok}[1]{\textcolor[rgb]{0.00,0.00,0.81}{#1}}
\newcommand{\DocumentationTok}[1]{\textcolor[rgb]{0.56,0.35,0.01}{\textbf{\textit{#1}}}}
\newcommand{\ErrorTok}[1]{\textcolor[rgb]{0.64,0.00,0.00}{\textbf{#1}}}
\newcommand{\ExtensionTok}[1]{#1}
\newcommand{\FloatTok}[1]{\textcolor[rgb]{0.00,0.00,0.81}{#1}}
\newcommand{\FunctionTok}[1]{\textcolor[rgb]{0.13,0.29,0.53}{\textbf{#1}}}
\newcommand{\ImportTok}[1]{#1}
\newcommand{\InformationTok}[1]{\textcolor[rgb]{0.56,0.35,0.01}{\textbf{\textit{#1}}}}
\newcommand{\KeywordTok}[1]{\textcolor[rgb]{0.13,0.29,0.53}{\textbf{#1}}}
\newcommand{\NormalTok}[1]{#1}
\newcommand{\OperatorTok}[1]{\textcolor[rgb]{0.81,0.36,0.00}{\textbf{#1}}}
\newcommand{\OtherTok}[1]{\textcolor[rgb]{0.56,0.35,0.01}{#1}}
\newcommand{\PreprocessorTok}[1]{\textcolor[rgb]{0.56,0.35,0.01}{\textit{#1}}}
\newcommand{\RegionMarkerTok}[1]{#1}
\newcommand{\SpecialCharTok}[1]{\textcolor[rgb]{0.81,0.36,0.00}{\textbf{#1}}}
\newcommand{\SpecialStringTok}[1]{\textcolor[rgb]{0.31,0.60,0.02}{#1}}
\newcommand{\StringTok}[1]{\textcolor[rgb]{0.31,0.60,0.02}{#1}}
\newcommand{\VariableTok}[1]{\textcolor[rgb]{0.00,0.00,0.00}{#1}}
\newcommand{\VerbatimStringTok}[1]{\textcolor[rgb]{0.31,0.60,0.02}{#1}}
\newcommand{\WarningTok}[1]{\textcolor[rgb]{0.56,0.35,0.01}{\textbf{\textit{#1}}}}
\usepackage{graphicx}
\makeatletter
\def\maxwidth{\ifdim\Gin@nat@width>\linewidth\linewidth\else\Gin@nat@width\fi}
\def\maxheight{\ifdim\Gin@nat@height>\textheight\textheight\else\Gin@nat@height\fi}
\makeatother
% Scale images if necessary, so that they will not overflow the page
% margins by default, and it is still possible to overwrite the defaults
% using explicit options in \includegraphics[width, height, ...]{}
\setkeys{Gin}{width=\maxwidth,height=\maxheight,keepaspectratio}
% Set default figure placement to htbp
\makeatletter
\def\fps@figure{htbp}
\makeatother
\setlength{\emergencystretch}{3em} % prevent overfull lines
\providecommand{\tightlist}{%
  \setlength{\itemsep}{0pt}\setlength{\parskip}{0pt}}
\setcounter{secnumdepth}{-\maxdimen} % remove section numbering
\ifLuaTeX
  \usepackage{selnolig}  % disable illegal ligatures
\fi
\IfFileExists{bookmark.sty}{\usepackage{bookmark}}{\usepackage{hyperref}}
\IfFileExists{xurl.sty}{\usepackage{xurl}}{} % add URL line breaks if available
\urlstyle{same}
\hypersetup{
  pdftitle={R Stats Tuto},
  pdfauthor={Pierre-Alexandre},
  hidelinks,
  pdfcreator={LaTeX via pandoc}}

\title{R Stats Tuto}
\author{Pierre-Alexandre}
\date{2023-10-11}

\begin{document}
\maketitle

\begin{center}\rule{0.5\linewidth}{0.5pt}\end{center}

\hypertarget{r-markdown}{%
\subsection{\texorpdfstring{\emph{R
Markdown}}{R Markdown}}\label{r-markdown}}

\emph{This is an R Markdown document. Markdown is a simple formatting
syntax for authoring HTML, PDF, and MS Word documents. For more details
on using R Markdown see \url{http://rmarkdown.rstudio.com} or the
YouTube tutorial (in french)
\url{https://www.youtube.com/watch?v=lFdB4fIAcLM}}

\emph{A cheat-sheet can be find here:
\url{https://rstudio.github.io/cheatsheets/html/rmarkdown.html?_gl=1*a65iq2*_ga*MTk3MDA5NzQ0MS4xNjk3MDYyMzE3*_ga_2C0WZ1JHG0*MTY5NzA2MjMxNi4xLjEuMTY5NzA2MzAzNy4wLjAuMA..}}

\hypertarget{preparation-of-the-work-space}{%
\subsection{\texorpdfstring{\emph{Preparation of the work
space}}{Preparation of the work space}}\label{preparation-of-the-work-space}}

\hypertarget{r-rstudio-install}{%
\subsubsection{\texorpdfstring{\emph{R \& RStudio
install}}{R \& RStudio install}}\label{r-rstudio-install}}

\emph{R \& the IDE RStudio can be install from this web page :
\url{https://posit.co/download/rstudio-desktop/}}

\begin{enumerate}
\def\labelenumi{\arabic{enumi}.}
\tightlist
\item
  \emph{Download then Install R}
\item
  \emph{Download then Install RStudio}
\end{enumerate}

\emph{Cheat-sheet of RStudio :
\url{https://rstudio.github.io/cheatsheets/html/rstudio-ide.html?_gl=1*1tpycou*_ga*MTk3MDA5NzQ0MS4xNjk3MDYyMzE3*_ga_2C0WZ1JHG0*MTY5NzA2MjMxNi4xLjEuMTY5NzA2MzAzNy4wLjAuMA..}}

\hypertarget{install-package}{%
\subsubsection{\texorpdfstring{\emph{Install
package:}}{Install package:}}\label{install-package}}

\emph{Packages can be install with the instruction
\texttt{install.packages()} (Don't forget the\texttt{"\ "}):\\
Example for the \texttt{ggplot2} package:}

\begin{Shaded}
\begin{Highlighting}[]
\CommentTok{\# Install from CRAN}
\FunctionTok{install.packages}\NormalTok{(}\StringTok{"ggplot2"}\NormalTok{)}
\end{Highlighting}
\end{Shaded}

\emph{To see all the packages which are installed use the code below:}

\begin{Shaded}
\begin{Highlighting}[]
\CommentTok{\# Installed packages}
\FunctionTok{installed.packages}\NormalTok{()}
\end{Highlighting}
\end{Shaded}

\textbf{Use a package}

After the installation of the desired package, it is neceesary to upload
it in R thanks to \texttt{library} command:

\begin{Shaded}
\begin{Highlighting}[]
\FunctionTok{library}\NormalTok{(}\StringTok{"ggplot2"}\NormalTok{, }\StringTok{"tibble"}\NormalTok{)}
\end{Highlighting}
\end{Shaded}

An other way consist to put all the library desired in a variable thank
to a vector (see below), then upload it with the function
\texttt{lapply} as it is showed below:

\begin{Shaded}
\begin{Highlighting}[]
\NormalTok{x }\OtherTok{\textless{}{-}} \FunctionTok{c}\NormalTok{(}\StringTok{"readr"}\NormalTok{,}\StringTok{"tibble"}\NormalTok{,}\StringTok{"tidyr"}\NormalTok{,}\StringTok{"ggplot2"}\NormalTok{,}\StringTok{"dplyr"}\NormalTok{,}\StringTok{"gridExtra"}\NormalTok{, }\StringTok{"pracma"}\NormalTok{, }\StringTok{"factoextra"}\NormalTok{, }\StringTok{"FactoMineR"}\NormalTok{, }\StringTok{"Cairo"}\NormalTok{, }\StringTok{"Rtsne"}\NormalTok{, }
       \StringTok{"colorspace"}\NormalTok{,}\StringTok{"dendextend"}\NormalTok{, }\StringTok{"RColorBrewer"}\NormalTok{, }\StringTok{"ggthemes"}\NormalTok{,}\StringTok{"ggpubr"}\NormalTok{, }\StringTok{"readxl"}\NormalTok{, }\StringTok{"pspline"}\NormalTok{, }\StringTok{"randomForest"}\NormalTok{, }\StringTok{"lattice"}\NormalTok{,}
       \StringTok{"caret"}\NormalTok{, }\StringTok{"LiblineaR"}\NormalTok{)}
\FunctionTok{lapply}\NormalTok{(x,require,}\AttributeTok{character.only =}\NormalTok{ T)}
\end{Highlighting}
\end{Shaded}

\begin{center}\rule{0.5\linewidth}{0.5pt}\end{center}

\hypertarget{the-basic}{%
\section{The basic}\label{the-basic}}

Good resources: \url{https://bookdown.org/ael/rexplor/}

\hypertarget{assignation-of-a-variable}{%
\subsection{Assignation of a
variable:}\label{assignation-of-a-variable}}

In R, the variable can be assigned with two symbol \texttt{\textless{}-}
OR \texttt{=}. But the first one is the most used.\\
To compile a ``list'' of variable, it is needed to create a vector
thanks to \texttt{c()} with \texttt{,} for the separation.\\
Then thanks to the funtion \texttt{class} you can see the class of the
variable.

\begin{Shaded}
\begin{Highlighting}[]
\NormalTok{x }\OtherTok{\textless{}{-}} \DecValTok{20}
\NormalTok{l }\OtherTok{\textless{}{-}} \FunctionTok{c}\NormalTok{(}\DecValTok{10}\NormalTok{, }\DecValTok{44}\NormalTok{, }\DecValTok{89}\NormalTok{)}
\NormalTok{s }\OtherTok{\textless{}{-}} \StringTok{"I am a character"}
\NormalTok{bo }\OtherTok{\textless{}{-}} \ConstantTok{TRUE} \CommentTok{\# a booleen}

\FunctionTok{paste}\NormalTok{(}\StringTok{"Variable x:"}\NormalTok{,x)}
\end{Highlighting}
\end{Shaded}

\begin{verbatim}
## [1] "Variable x: 20"
\end{verbatim}

\begin{Shaded}
\begin{Highlighting}[]
\FunctionTok{class}\NormalTok{(x)}
\end{Highlighting}
\end{Shaded}

\begin{verbatim}
## [1] "numeric"
\end{verbatim}

\begin{Shaded}
\begin{Highlighting}[]
\FunctionTok{paste}\NormalTok{(}\StringTok{"Variable l:"}\NormalTok{,l)}
\end{Highlighting}
\end{Shaded}

\begin{verbatim}
## [1] "Variable l: 10" "Variable l: 44" "Variable l: 89"
\end{verbatim}

\begin{Shaded}
\begin{Highlighting}[]
\FunctionTok{class}\NormalTok{(l)}
\end{Highlighting}
\end{Shaded}

\begin{verbatim}
## [1] "numeric"
\end{verbatim}

\begin{Shaded}
\begin{Highlighting}[]
\FunctionTok{paste}\NormalTok{(}\StringTok{"Variable s:"}\NormalTok{, s)}
\end{Highlighting}
\end{Shaded}

\begin{verbatim}
## [1] "Variable s: I am a character"
\end{verbatim}

\begin{Shaded}
\begin{Highlighting}[]
\FunctionTok{class}\NormalTok{(s)}
\end{Highlighting}
\end{Shaded}

\begin{verbatim}
## [1] "character"
\end{verbatim}

\begin{Shaded}
\begin{Highlighting}[]
\FunctionTok{paste}\NormalTok{(}\StringTok{"Variable s:"}\NormalTok{, bo)}
\end{Highlighting}
\end{Shaded}

\begin{verbatim}
## [1] "Variable s: TRUE"
\end{verbatim}

\begin{Shaded}
\begin{Highlighting}[]
\FunctionTok{class}\NormalTok{(bo)}
\end{Highlighting}
\end{Shaded}

\begin{verbatim}
## [1] "logical"
\end{verbatim}

\emph{Remarks}\\
The utilization of the vector \texttt{c()} do not create a real list but
a \textbf{vector}.\\
For a list, it is needed to call the function \texttt{list}:

\begin{Shaded}
\begin{Highlighting}[]
\NormalTok{li }\OtherTok{\textless{}{-}} \FunctionTok{list}\NormalTok{(}\DecValTok{10}\NormalTok{, }\DecValTok{44}\NormalTok{, }\DecValTok{89}\NormalTok{)}
\NormalTok{li}
\end{Highlighting}
\end{Shaded}

\begin{verbatim}
## [[1]]
## [1] 10
## 
## [[2]]
## [1] 44
## 
## [[3]]
## [1] 89
\end{verbatim}

\begin{Shaded}
\begin{Highlighting}[]
\FunctionTok{class}\NormalTok{(li)}
\end{Highlighting}
\end{Shaded}

\begin{verbatim}
## [1] "list"
\end{verbatim}

The vector can not create missed value because all that is useless is
not created. Instead of, the list permit the creation with null value:

\begin{Shaded}
\begin{Highlighting}[]
\NormalTok{vector }\OtherTok{\textless{}{-}} \FunctionTok{c}\NormalTok{(}\DecValTok{1}\NormalTok{, }\DecValTok{5}\NormalTok{, }\ConstantTok{NULL}\NormalTok{)}
\NormalTok{liste }\OtherTok{\textless{}{-}} \FunctionTok{list}\NormalTok{(}\DecValTok{1}\NormalTok{, }\DecValTok{5}\NormalTok{, }\StringTok{"u"}\NormalTok{, }\ConstantTok{NULL}\NormalTok{)}
\NormalTok{vector}
\end{Highlighting}
\end{Shaded}

\begin{verbatim}
## [1] 1 5
\end{verbatim}

\begin{Shaded}
\begin{Highlighting}[]
\NormalTok{liste}
\end{Highlighting}
\end{Shaded}

\begin{verbatim}
## [[1]]
## [1] 1
## 
## [[2]]
## [1] 5
## 
## [[3]]
## [1] "u"
## 
## [[4]]
## NULL
\end{verbatim}

The command \texttt{paste()} is used to concatenate string and variable.
A space is automatically include as separator. To change it, add the
argument \texttt{,\ sep=""} with no space or other symbol. It is
possible to used also \texttt{paste0}.

For any help, use the command \texttt{help()}or the symbol \texttt{?}.
Example:

\begin{Shaded}
\begin{Highlighting}[]
\FunctionTok{help}\NormalTok{(class)}
\NormalTok{?ggplot2}
\end{Highlighting}
\end{Shaded}

To remove an object: \texttt{rm()}. At the begining of a
session/project, it can be usefull to clean all the data which was kept
by RSudio in its memory:

\begin{Shaded}
\begin{Highlighting}[]
\FunctionTok{rm}\NormalTok{(}\AttributeTok{list =} \FunctionTok{ls}\NormalTok{())}
\end{Highlighting}
\end{Shaded}

\hypertarget{dataset-manipulation}{%
\subsection{Dataset Manipulation}\label{dataset-manipulation}}

\hypertarget{importation-of-csv-file}{%
\subsubsection{Importation of csv file}\label{importation-of-csv-file}}

csv files are directly import as a \texttt{data.frame}. A data set can
be displayed thanks to the command \texttt{View}.

\begin{Shaded}
\begin{Highlighting}[]
\NormalTok{smp }\OtherTok{\textless{}{-}} \FunctionTok{read.csv2}\NormalTok{(}\StringTok{"\textasciitilde{}/code/R/FUN/fichier 1/smp1.csv"}\NormalTok{)}
\FunctionTok{View}\NormalTok{(smp) }\CommentTok{\# With upper V }
\end{Highlighting}
\end{Shaded}

\hypertarget{some-basic-manipulations}{%
\subsubsection{Some basic
manipulations}\label{some-basic-manipulations}}

To see the 6 first observations for the all variables, use the command
\texttt{head()}. Use \texttt{\$} to access to a variable of a data set:

\begin{Shaded}
\begin{Highlighting}[]
\CommentTok{\# Utilisation of the Iris data set, directly included in R}
\FunctionTok{class}\NormalTok{(iris)}
\end{Highlighting}
\end{Shaded}

\begin{verbatim}
## [1] "data.frame"
\end{verbatim}

\begin{Shaded}
\begin{Highlighting}[]
\FunctionTok{head}\NormalTok{(iris)}
\end{Highlighting}
\end{Shaded}

\begin{verbatim}
##   Sepal.Length Sepal.Width Petal.Length Petal.Width Species
## 1          5.1         3.5          1.4         0.2  setosa
## 2          4.9         3.0          1.4         0.2  setosa
## 3          4.7         3.2          1.3         0.2  setosa
## 4          4.6         3.1          1.5         0.2  setosa
## 5          5.0         3.6          1.4         0.2  setosa
## 6          5.4         3.9          1.7         0.4  setosa
\end{verbatim}

\begin{Shaded}
\begin{Highlighting}[]
\FunctionTok{head}\NormalTok{(iris}\SpecialCharTok{$}\NormalTok{Sepal.Length)}
\end{Highlighting}
\end{Shaded}

\begin{verbatim}
## [1] 5.1 4.9 4.7 4.6 5.0 5.4
\end{verbatim}

The command \texttt{which}give the observation who respect the
condition:

\begin{itemize}
\tightlist
\item
  \texttt{==} for an equivalence
\item
  \texttt{!\textgreater{}} for a difference
\item
  \texttt{\textless{}=} / \texttt{\textgreater{}=} less/greater than or
  equal to
\item
  \texttt{\textless{}}/ \texttt{\textgreater{}} strictly less/greater
  than
\item
  \texttt{\&} / \texttt{\textbar{}} AND / OR
\item
  \texttt{isTRUE(x)} test if X is TRUE
\item
  \texttt{na.rm=TRUE} to remove the empty values
\end{itemize}

\begin{Shaded}
\begin{Highlighting}[]
\FunctionTok{which}\NormalTok{(iris}\SpecialCharTok{$}\NormalTok{Species }\SpecialCharTok{!=} \StringTok{"setosa"}\NormalTok{)}
\end{Highlighting}
\end{Shaded}

\begin{verbatim}
##   [1]  51  52  53  54  55  56  57  58  59  60  61  62  63  64  65  66  67  68
##  [19]  69  70  71  72  73  74  75  76  77  78  79  80  81  82  83  84  85  86
##  [37]  87  88  89  90  91  92  93  94  95  96  97  98  99 100 101 102 103 104
##  [55] 105 106 107 108 109 110 111 112 113 114 115 116 117 118 119 120 121 122
##  [73] 123 124 125 126 127 128 129 130 131 132 133 134 135 136 137 138 139 140
##  [91] 141 142 143 144 145 146 147 148 149 150
\end{verbatim}

The command \texttt{table} group and count each observation for a given
variable. The command
\texttt{subset(data.frame,\ condition,\ c(vector\ of\ variable))}, take
the observation who respect the condition and keep only the variables
asked.

\begin{Shaded}
\begin{Highlighting}[]
\FunctionTok{table}\NormalTok{(iris}\SpecialCharTok{$}\NormalTok{Species)}
\end{Highlighting}
\end{Shaded}

\begin{verbatim}
## 
##     setosa versicolor  virginica 
##         50         50         50
\end{verbatim}

\begin{Shaded}
\begin{Highlighting}[]
\FunctionTok{table}\NormalTok{(iris}\SpecialCharTok{$}\NormalTok{Species }\SpecialCharTok{!=} \StringTok{"setosa"}\NormalTok{)}
\end{Highlighting}
\end{Shaded}

\begin{verbatim}
## 
## FALSE  TRUE 
##    50   100
\end{verbatim}

\begin{Shaded}
\begin{Highlighting}[]
\FunctionTok{table}\NormalTok{(}\FunctionTok{subset}\NormalTok{(iris, Species }\SpecialCharTok{==} \StringTok{"setosa"}\NormalTok{)}\SpecialCharTok{$}\NormalTok{Petal.Length }\SpecialCharTok{\textgreater{}} \FloatTok{1.4}\NormalTok{)}
\end{Highlighting}
\end{Shaded}

\begin{verbatim}
## 
## FALSE  TRUE 
##    24    26
\end{verbatim}

\begin{Shaded}
\begin{Highlighting}[]
\FunctionTok{head}\NormalTok{(}\FunctionTok{subset}\NormalTok{(iris, Species }\SpecialCharTok{==} \StringTok{"setosa"}\NormalTok{, }\FunctionTok{c}\NormalTok{(Sepal.Length, Petal.Length)))}
\end{Highlighting}
\end{Shaded}

\begin{verbatim}
##   Sepal.Length Petal.Length
## 1          5.1          1.4
## 2          4.9          1.4
## 3          4.7          1.3
## 4          4.6          1.5
## 5          5.0          1.4
## 6          5.4          1.7
\end{verbatim}

\hypertarget{make-a-tibble-instead-of-data.frame}{%
\subsubsection{\texorpdfstring{Make a \emph{tibble} instead of
\emph{data.frame}}{Make a tibble instead of data.frame}}\label{make-a-tibble-instead-of-data.frame}}

Upload the library \texttt{tibble}. This one allowed the creation of a
\emph{tibble} (table) that it has the same purpose than a
\emph{data.frame} but with more restriction. The \emph{tibble} must to
have the same size for each column.

\begin{Shaded}
\begin{Highlighting}[]
\FunctionTok{library}\NormalTok{(tibble)}
\CommentTok{\# creation of the dataset}
\CommentTok{\# m1 a list with Null values}
\NormalTok{m1}\OtherTok{\textless{}{-}}\FunctionTok{list}\NormalTok{(}\FloatTok{1.311}\NormalTok{,}\FloatTok{1.287}\NormalTok{,}\FloatTok{1.293}\NormalTok{,}\FloatTok{1.308}\NormalTok{,}\FloatTok{1.291}\NormalTok{,}\FloatTok{1.300}\NormalTok{,}\FloatTok{1.274}\NormalTok{,}\FloatTok{1.287}\NormalTok{)}
\NormalTok{m1 }\OtherTok{\textless{}{-}} \FunctionTok{append}\NormalTok{(m1, }\FunctionTok{vector}\NormalTok{(}\StringTok{"list"}\NormalTok{,}\DecValTok{5}\NormalTok{)) }\CommentTok{\# append 5 element NULL contain in a list}
\NormalTok{m2}\OtherTok{\textless{}{-}}\FunctionTok{c}\NormalTok{(}\FloatTok{1.298}\NormalTok{,}\FloatTok{1.309}\NormalTok{,}\FloatTok{1.293}\NormalTok{,}\FloatTok{1.251}\NormalTok{,}\FloatTok{1.338}\NormalTok{,}\FloatTok{1.302}\NormalTok{,}\FloatTok{1.270}\NormalTok{,}\FloatTok{1.339}\NormalTok{,}\FloatTok{1.346}\NormalTok{,}\FloatTok{1.292}\NormalTok{,}\FloatTok{1.291}\NormalTok{,}\FloatTok{1.321}\NormalTok{,}\FloatTok{1.285}\NormalTok{)}
\NormalTok{Si}\OtherTok{\textless{}{-}}\FunctionTok{tibble}\NormalTok{(}\AttributeTok{method1=}\NormalTok{m1,}\AttributeTok{method2=}\NormalTok{m2)}
\end{Highlighting}
\end{Shaded}

\textbf{Use the functional programming}

The functional programming is a paradigm of building computer program
with a declarative type. In this case, the programs are constructed by
applying and composing functions. In R the \texttt{tibble} library (and
other \emph{tixxx}) allow this kind programming with the following
construction \texttt{variable\ \%\textgreater{}\%\ function()}. The
functional is a better way to write a comprehensive workflow:

\begin{Shaded}
\begin{Highlighting}[]
\CommentTok{\# recover a list containing the values of method 1 padded with NULL values then converted to a vector, without NULL values}
\NormalTok{m1}\OtherTok{\textless{}{-}}\NormalTok{Si}\SpecialCharTok{$}\NormalTok{method1 }\SpecialCharTok{\%\textgreater{}\%} \FunctionTok{unlist}\NormalTok{()}
\NormalTok{m1}
\end{Highlighting}
\end{Shaded}

\begin{verbatim}
## [1] 1.311 1.287 1.293 1.308 1.291 1.300 1.274 1.287
\end{verbatim}

\begin{Shaded}
\begin{Highlighting}[]
\FunctionTok{class}\NormalTok{(m1)}
\end{Highlighting}
\end{Shaded}

\begin{verbatim}
## [1] "numeric"
\end{verbatim}

\hypertarget{manipulation}{%
\subsubsection{Manipulation}\label{manipulation}}

The libraries \texttt{tidyr} \& \texttt{dplyr} are useful to manipulate
a data set.

For example, in the first table, it is supposed that each samples have
been measured with the both two methods:

\begin{itemize}
\tightlist
\item
  sample 1 : value m1 / value m2
\item
  sample 2 : value m1 / value m2
\item
  \ldots{}
\end{itemize}

In this case, measurements have been performed independently of the
samples. So the first table does not work. Here it will be better that
each value receive tag corresponding to its origin (method1 or method2).
The code below propose this kind of table:

\begin{Shaded}
\begin{Highlighting}[]
\FunctionTok{library}\NormalTok{(dplyr, tidyr)}
\end{Highlighting}
\end{Shaded}

\begin{verbatim}
## 
## Attachement du package : 'dplyr'
\end{verbatim}

\begin{verbatim}
## Les objets suivants sont masqués depuis 'package:stats':
## 
##     filter, lag
\end{verbatim}

\begin{verbatim}
## Les objets suivants sont masqués depuis 'package:base':
## 
##     intersect, setdiff, setequal, union
\end{verbatim}

\begin{Shaded}
\begin{Highlighting}[]
\NormalTok{Si2}\OtherTok{\textless{}{-}}\FunctionTok{tibble}\NormalTok{(}\AttributeTok{method=}\FunctionTok{c}\NormalTok{(}\FunctionTok{rep}\NormalTok{(}\StringTok{"method1"}\NormalTok{,}\DecValTok{8}\NormalTok{),}\FunctionTok{rep}\NormalTok{(}\StringTok{"method2"}\NormalTok{,}\DecValTok{13}\NormalTok{)),}\AttributeTok{values=}\FunctionTok{c}\NormalTok{(m1,m2))}

\CommentTok{\# Let\textquotesingle{}s simulate a randomized acquisition of the dataset}
\NormalTok{Si2}\OtherTok{\textless{}{-}}\NormalTok{Si2[}\FunctionTok{sample}\NormalTok{(}\FunctionTok{nrow}\NormalTok{(Si2)),] }\CommentTok{\# Indexes are generated by random pick, the tibble is shuffled}
\FunctionTok{head}\NormalTok{(Si2)}
\end{Highlighting}
\end{Shaded}

\begin{verbatim}
## # A tibble: 6 x 2
##   method  values
##   <chr>    <dbl>
## 1 method1   1.29
## 2 method2   1.34
## 3 method1   1.27
## 4 method2   1.27
## 5 method2   1.32
## 6 method1   1.29
\end{verbatim}

\textbf{Play with \texttt{tidyr} \& \texttt{dplyr} libraries and
functional programming}

\begin{Shaded}
\begin{Highlighting}[]
\NormalTok{m1b}\OtherTok{\textless{}{-}}\NormalTok{Si2 }\SpecialCharTok{\%\textgreater{}\%} \FunctionTok{filter}\NormalTok{(method }\SpecialCharTok{==} \StringTok{"method1"}\NormalTok{) }\SpecialCharTok{\%\textgreater{}\%} \FunctionTok{select}\NormalTok{(values) }\SpecialCharTok{\%\textgreater{}\%} \FunctionTok{unlist}\NormalTok{()}
\NormalTok{m1}
\end{Highlighting}
\end{Shaded}

\begin{verbatim}
## [1] 1.311 1.287 1.293 1.308 1.291 1.300 1.274 1.287
\end{verbatim}

\begin{Shaded}
\begin{Highlighting}[]
\NormalTok{m1b}
\end{Highlighting}
\end{Shaded}

\begin{verbatim}
## values1 values2 values3 values4 values5 values6 values7 values8 
##   1.287   1.274   1.291   1.308   1.311   1.287   1.293   1.300
\end{verbatim}

Cheat-sheet for more information on those libraries:

\url{https://www.rstudio.com/wp-content/uploads/2015/02/data-wrangling-cheatsheet.pdf}

\hypertarget{statistics-operations}{%
\subsection{Statistics operations}\label{statistics-operations}}

A rapid view of the global statistic descriptor with the function
\texttt{summary}:

\begin{Shaded}
\begin{Highlighting}[]
\FunctionTok{summary}\NormalTok{(iris)}
\end{Highlighting}
\end{Shaded}

\begin{verbatim}
##   Sepal.Length    Sepal.Width     Petal.Length    Petal.Width   
##  Min.   :4.300   Min.   :2.000   Min.   :1.000   Min.   :0.100  
##  1st Qu.:5.100   1st Qu.:2.800   1st Qu.:1.600   1st Qu.:0.300  
##  Median :5.800   Median :3.000   Median :4.350   Median :1.300  
##  Mean   :5.843   Mean   :3.057   Mean   :3.758   Mean   :1.199  
##  3rd Qu.:6.400   3rd Qu.:3.300   3rd Qu.:5.100   3rd Qu.:1.800  
##  Max.   :7.900   Max.   :4.400   Max.   :6.900   Max.   :2.500  
##        Species  
##  setosa    :50  
##  versicolor:50  
##  virginica :50  
##                 
##                 
## 
\end{verbatim}

\emph{Note: iris is a dataset which is already include in R}

\hypertarget{basic-operations}{%
\subsubsection{\texorpdfstring{\emph{Basic
operations:}}{Basic operations:}}\label{basic-operations}}

\begin{Shaded}
\begin{Highlighting}[]
\FunctionTok{sum}\NormalTok{(iris}\SpecialCharTok{$}\NormalTok{Sepal.Length) }\CommentTok{\# Addition of all the observation in the variable}
\end{Highlighting}
\end{Shaded}

\begin{verbatim}
## [1] 876.5
\end{verbatim}

\begin{Shaded}
\begin{Highlighting}[]
\FunctionTok{mean}\NormalTok{(iris}\SpecialCharTok{$}\NormalTok{Sepal.Length) }\CommentTok{\# Average}
\end{Highlighting}
\end{Shaded}

\begin{verbatim}
## [1] 5.843333
\end{verbatim}

\begin{Shaded}
\begin{Highlighting}[]
\FunctionTok{sd}\NormalTok{(iris}\SpecialCharTok{$}\NormalTok{Sepal.Length) }\CommentTok{\# Standard deviation}
\end{Highlighting}
\end{Shaded}

\begin{verbatim}
## [1] 0.8280661
\end{verbatim}

\begin{Shaded}
\begin{Highlighting}[]
\NormalTok{q }\OtherTok{\textless{}{-}} \FunctionTok{quantile}\NormalTok{(iris}\SpecialCharTok{$}\NormalTok{Sepal.Length) }\CommentTok{\# Catch all the quantiles }
\NormalTok{q}
\end{Highlighting}
\end{Shaded}

\begin{verbatim}
##   0%  25%  50%  75% 100% 
##  4.3  5.1  5.8  6.4  7.9
\end{verbatim}

\begin{Shaded}
\begin{Highlighting}[]
\NormalTok{q[}\DecValTok{1}\NormalTok{]}
\end{Highlighting}
\end{Shaded}

\begin{verbatim}
##  0% 
## 4.3
\end{verbatim}

\begin{Shaded}
\begin{Highlighting}[]
\FunctionTok{length}\NormalTok{(iris}\SpecialCharTok{$}\NormalTok{Sepal.Length) }\CommentTok{\# Number of observation in the variable}
\end{Highlighting}
\end{Shaded}

\begin{verbatim}
## [1] 150
\end{verbatim}

\hypertarget{confidence-interval}{%
\subsubsection{\texorpdfstring{\emph{Confidence
interval}}{Confidence interval}}\label{confidence-interval}}

The confidence interval at 95\% : \(IC=\bar{x}\pm1,96*se\) with:

\begin{itemize}
\tightlist
\item
  \(se\) the standard error ==\textgreater{} the standard deviation
  \(\sigma\) of the mean
\item
  \(se=\frac{\sigma}{\sqrt{n-1}}\) with \(n\) the number of observations
\item
  The \(1.96\) correspond to a quantile \(\gamma\) of a normal student
  law for a confidence \(\alpha\) of 95\% given by
  \(\gamma=\frac{1-\alpha}{2} = \frac{1-0.95}{2}=0.025\)
\item
  R can calculate \(\gamma\) with: \texttt{qnorm(y)} \&
  \texttt{qnorm(1-y)}\\
\end{itemize}

\begin{Shaded}
\begin{Highlighting}[]
\NormalTok{y }\OtherTok{\textless{}{-}}\NormalTok{ (}\DecValTok{1}\FloatTok{{-}0.95}\NormalTok{)}\SpecialCharTok{/}\DecValTok{2}
\NormalTok{y}
\end{Highlighting}
\end{Shaded}

\begin{verbatim}
## [1] 0.025
\end{verbatim}

\begin{Shaded}
\begin{Highlighting}[]
\FunctionTok{signif}\NormalTok{(}\FunctionTok{qnorm}\NormalTok{(y),}\DecValTok{3}\NormalTok{)}
\end{Highlighting}
\end{Shaded}

\begin{verbatim}
## [1] -1.96
\end{verbatim}

\begin{Shaded}
\begin{Highlighting}[]
\FunctionTok{signif}\NormalTok{(}\FunctionTok{qnorm}\NormalTok{(}\DecValTok{1}\SpecialCharTok{{-}}\NormalTok{y),}\DecValTok{3}\NormalTok{)}
\end{Highlighting}
\end{Shaded}

\begin{verbatim}
## [1] 1.96
\end{verbatim}

\emph{The standard error is the standard deviation of the mean of
distributions and the standard deviation is for the individual
distribution of a normal law.}

\begin{Shaded}
\begin{Highlighting}[]
\NormalTok{x }\OtherTok{\textless{}{-}}\NormalTok{ iris}\SpecialCharTok{$}\NormalTok{Sepal.Length}
\NormalTok{x\_N}\OtherTok{\textless{}{-}}\FunctionTok{length}\NormalTok{(x)}
\NormalTok{x\_sd}\OtherTok{\textless{}{-}}\FunctionTok{sd}\NormalTok{(x)}
\NormalTok{x\_mu}\OtherTok{\textless{}{-}}\FunctionTok{mean}\NormalTok{(x)}
\NormalTok{x\_se}\OtherTok{\textless{}{-}}\NormalTok{x\_sd}\SpecialCharTok{/}\FunctionTok{sqrt}\NormalTok{(x\_N)}
\NormalTok{x\_ICmin}\OtherTok{\textless{}{-}}\NormalTok{x\_mu}\SpecialCharTok{+}\NormalTok{x\_se}\SpecialCharTok{*}\FunctionTok{qnorm}\NormalTok{(}\FloatTok{0.025}\NormalTok{)}
\NormalTok{x\_ICmax}\OtherTok{\textless{}{-}}\NormalTok{x\_mu}\SpecialCharTok{+}\NormalTok{x\_se}\SpecialCharTok{*}\FunctionTok{qnorm}\NormalTok{(}\FloatTok{0.975}\NormalTok{)}
\FunctionTok{paste}\NormalTok{(}\FunctionTok{signif}\NormalTok{(x\_ICmin,}\DecValTok{3}\NormalTok{),}\StringTok{"\textless{}"}\NormalTok{,}\FunctionTok{signif}\NormalTok{(x\_mu,}\DecValTok{3}\NormalTok{),}\StringTok{"\textless{}"}\NormalTok{,}\FunctionTok{signif}\NormalTok{(x\_ICmax,}\DecValTok{3}\NormalTok{))}
\end{Highlighting}
\end{Shaded}

\begin{verbatim}
## [1] "5.71 < 5.84 < 5.98"
\end{verbatim}

The command \texttt{signif(variable,\ number)} print the
\texttt{variable} with the \texttt{number} of significant value asked.

\begin{center}\rule{0.5\linewidth}{0.5pt}\end{center}

\hypertarget{evaluation-of-probability-law}{%
\section{Evaluation of probability
law}\label{evaluation-of-probability-law}}

With R, it is easy to evaluate a probability law:

\begin{itemize}
\tightlist
\item
  {\textbf{r}}mylaw ==\textgreater{} Simulate mylaw
\item
  {\textbf{d}}mylaw ==\textgreater{} Calculate the density of mylaw
\item
  {\textbf{p}}mylaw ==\textgreater{} Calculate the cumulative of density
  function (CDF) of mylaw
\item
  {\textbf{q}}mylaw ==\textgreater{} Calculate the quantile of mylaw
\end{itemize}

Below the 3 mains laws:

\hypertarget{normal-law-gauss-law}{%
\subsection{\texorpdfstring{\emph{Normal Law (Gauss
Law)}}{Normal Law (Gauss Law)}}\label{normal-law-gauss-law}}

\begin{Shaded}
\begin{Highlighting}[]
\NormalTok{my\_mu}\OtherTok{\textless{}{-}}\DecValTok{20} \CommentTok{\# mean value of the normal law}
\NormalTok{my\_sg}\OtherTok{\textless{}{-}}\FunctionTok{sqrt}\NormalTok{(}\DecValTok{20}\NormalTok{) }\CommentTok{\# std dev of the normal law}
\FunctionTok{dnorm}\NormalTok{(}\DecValTok{20}\NormalTok{,}\AttributeTok{mean=}\NormalTok{my\_mu,}\AttributeTok{sd=}\NormalTok{my\_sg) }\CommentTok{\# point evaluation of the density at x=20 of a normal law centered at my\_mu and width sigma=my\_sg}
\end{Highlighting}
\end{Shaded}

\begin{verbatim}
## [1] 0.08920621
\end{verbatim}

\begin{Shaded}
\begin{Highlighting}[]
\FunctionTok{pnorm}\NormalTok{(}\DecValTok{20}\NormalTok{,}\AttributeTok{mean=}\NormalTok{my\_mu,}\AttributeTok{sd=}\NormalTok{my\_sg) }\CommentTok{\# point evaluation of the CDF at q=20 of this same normal distribution}
\end{Highlighting}
\end{Shaded}

\begin{verbatim}
## [1] 0.5
\end{verbatim}

\begin{Shaded}
\begin{Highlighting}[]
\FunctionTok{qnorm}\NormalTok{(}\FloatTok{0.05}\NormalTok{,}\AttributeTok{mean=}\NormalTok{my\_mu,}\AttributeTok{sd=}\NormalTok{my\_sg) }\CommentTok{\# quantile of this normal distribution considering a probability p=0.05}
\end{Highlighting}
\end{Shaded}

\begin{verbatim}
## [1] 12.64399
\end{verbatim}

\begin{Shaded}
\begin{Highlighting}[]
\FunctionTok{rnorm}\NormalTok{(}\DecValTok{10}\NormalTok{,}\AttributeTok{mean=}\NormalTok{my\_mu,}\AttributeTok{sd=}\NormalTok{my\_sg) }\CommentTok{\# generate 10 random numbers following this normal distribution}
\end{Highlighting}
\end{Shaded}

\begin{verbatim}
##  [1] 23.33057 22.26138 18.38888 23.40127 15.80321 12.41356 13.25710 21.93364
##  [9] 21.01006 17.88063
\end{verbatim}

\begin{Shaded}
\begin{Highlighting}[]
\NormalTok{alpha}\OtherTok{=}\FloatTok{0.0001} \CommentTok{\# kind of "type I risk", the part of the plot I want to hide on the left/right sides.}
\NormalTok{xmin}\OtherTok{=}\FunctionTok{qnorm}\NormalTok{(alpha,}\AttributeTok{mean=}\NormalTok{my\_mu,}\AttributeTok{sd=}\NormalTok{my\_sg) }\CommentTok{\# left value of the range to plot the distribution}
\NormalTok{xmax}\OtherTok{=}\FunctionTok{qnorm}\NormalTok{(}\DecValTok{1}\SpecialCharTok{{-}}\NormalTok{alpha,}\AttributeTok{mean=}\NormalTok{my\_mu,}\AttributeTok{sd=}\NormalTok{my\_sg) }\CommentTok{\# right value of the range to plot the distribution}
\end{Highlighting}
\end{Shaded}

\hypertarget{binomial-law}{%
\subsection{\texorpdfstring{\emph{Binomial
law}}{Binomial law}}\label{binomial-law}}

The Binary law can be positive or negative, but in the general cases, it
is only the positive which is used. This is the law which is followed by
a randomization selection when there are \textbf{ONLY TWO} possibility
of results: A or B, alive or died. The probability to obtained each
evenement must be constant (but not necessary equal). The probability to
obtain the A event will be \(p\) and B will be \(q=(1-p)\). So, with
\(n\) selection, the probability to obtain k event of A
\(\mathbb{P}(X=k)\) will follow the binomial law \(B(n,p)\). The mean
\(\mu\) correspond to the expectation \(\mathbb{E}\) (\emph{espérance in
french}) and will be give by \(\mu=np\). the variance is given by
\(var=npq\).

\begin{Shaded}
\begin{Highlighting}[]
\NormalTok{n}\OtherTok{\textless{}{-}}\DecValTok{30} \CommentTok{\# population}
\NormalTok{p}\OtherTok{\textless{}{-}}\FloatTok{0.1} \CommentTok{\# probability of success}
\FunctionTok{dbinom}\NormalTok{(}\AttributeTok{x=}\DecValTok{1}\NormalTok{,}\AttributeTok{size=}\NormalTok{n,}\AttributeTok{prob=}\NormalTok{p) }\CommentTok{\# Probability that I win exactly 1 time in 30 plays}
\end{Highlighting}
\end{Shaded}

\begin{verbatim}
## [1] 0.1413039
\end{verbatim}

\begin{Shaded}
\begin{Highlighting}[]
\FunctionTok{pbinom}\NormalTok{(}\AttributeTok{q=}\DecValTok{2}\NormalTok{,}\AttributeTok{size=}\NormalTok{n,}\AttributeTok{prob=}\NormalTok{p) }\CommentTok{\# CDF, probability to win 0,1 or 2 times in 30 plays}
\end{Highlighting}
\end{Shaded}

\begin{verbatim}
## [1] 0.4113512
\end{verbatim}

\begin{Shaded}
\begin{Highlighting}[]
\FunctionTok{qbinom}\NormalTok{(}\AttributeTok{p=}\FloatTok{0.5}\NormalTok{,}\AttributeTok{size=}\NormalTok{n,}\AttributeTok{prob=}\NormalTok{p) }\CommentTok{\# Quantile of 0.5 from the binomial law}
\end{Highlighting}
\end{Shaded}

\begin{verbatim}
## [1] 3
\end{verbatim}

\begin{Shaded}
\begin{Highlighting}[]
\FunctionTok{rbinom}\NormalTok{(}\AttributeTok{n=}\DecValTok{10}\NormalTok{,}\AttributeTok{size=}\NormalTok{n,}\AttributeTok{prob=}\NormalTok{p) }\CommentTok{\# Sample of size 10 from the binomial law}
\end{Highlighting}
\end{Shaded}

\begin{verbatim}
##  [1] 1 4 2 3 2 5 4 3 3 4
\end{verbatim}

The formula of the binomial law is:

\[\mathbb{P}(X = k) = \binom{n}{k} p^k (1-p)^{n-k}\]

Where \(\binom{n}{k}\) is the binomial coefficient (n choose k) and can
be calculate with the command \texttt{choose(n,k)} or by
\(\binom{n}{k} =C_n^{k}= \frac{n!}{k!(n-k)!}\)

\hypertarget{poisson-law}{%
\subsection{\texorpdfstring{\emph{Poisson
law}}{Poisson law}}\label{poisson-law}}

The Poisson law is used to described the probability of a rare event in
an interval of time. It is an approximation of the binomial law when the
when \(p\) tends to zero (very law probability) and \(n\) tends to
\(\infty\) (a lot of repetition). This law take the parameter
\(\lambda = np\) and traduce the the expectation \(\mathbb{E}\) and the
variance (which is logical since
\(p\rightarrow0\Rightarrow q\rightarrow1\) in the binomial law). That's
why the Poisson law is define by \(P(\lambda)\) and represent the mean
of the probability where the rare event \(k\) is realized.

\begin{Shaded}
\begin{Highlighting}[]
\NormalTok{n}\OtherTok{\textless{}{-}}\DecValTok{30} \CommentTok{\# number of observation}
\NormalTok{m }\OtherTok{\textless{}{-}} \DecValTok{20} \CommentTok{\#  Poisson law parameter}
\FunctionTok{dpois}\NormalTok{(}\AttributeTok{x=}\DecValTok{1}\NormalTok{, }\AttributeTok{lambda =}\NormalTok{m ) }\CommentTok{\#}
\end{Highlighting}
\end{Shaded}

\begin{verbatim}
## [1] 4.122307e-08
\end{verbatim}

\begin{Shaded}
\begin{Highlighting}[]
\FunctionTok{ppois}\NormalTok{(}\AttributeTok{q=}\DecValTok{2}\NormalTok{, }\AttributeTok{lambda =}\NormalTok{m ) }\CommentTok{\# }
\end{Highlighting}
\end{Shaded}

\begin{verbatim}
## [1] 4.55515e-07
\end{verbatim}

\begin{Shaded}
\begin{Highlighting}[]
\FunctionTok{qpois}\NormalTok{(}\AttributeTok{p=}\FloatTok{0.5}\NormalTok{, }\AttributeTok{lambda =}\NormalTok{m ) }\CommentTok{\# }
\end{Highlighting}
\end{Shaded}

\begin{verbatim}
## [1] 20
\end{verbatim}

\begin{Shaded}
\begin{Highlighting}[]
\FunctionTok{rpois}\NormalTok{(}\AttributeTok{n=}\NormalTok{n, }\AttributeTok{lambda =}\NormalTok{m ) }\CommentTok{\# }
\end{Highlighting}
\end{Shaded}

\begin{verbatim}
##  [1] 19 17 19 21 22 17 23 20 25 12 17 16 22 25 16 21 19 22 26 18 27 20 21 12 21
## [26] 25 24 20 20 11
\end{verbatim}

The formaula of the Poisson law is:
\[\mathbb{P}(X = k) = \frac{e^{-\lambda} \lambda^k}{k!}\]

\emph{Note: factorial calculation in R:}

\begin{Shaded}
\begin{Highlighting}[]
\FunctionTok{factorial}\NormalTok{(}\DecValTok{5}\NormalTok{)}
\end{Highlighting}
\end{Shaded}

\begin{verbatim}
## [1] 120
\end{verbatim}

\begin{center}\rule{0.5\linewidth}{0.5pt}\end{center}

\hypertarget{graphics}{%
\section{Graphics}\label{graphics}}

Some graphics can be use in R without package. But they are limited and
don't have a lot possibility to manage them. Those graphs can be useful
for a first and fast exploration of the data but for a presentation, it
must be interested to create more complex/beautful graphs. That why it
is better to work with the package \texttt{ggplot2}.

First of all, install the package \texttt{ggplot2} then upload it in R
thanks to \texttt{library} command:

\begin{Shaded}
\begin{Highlighting}[]
\FunctionTok{library}\NormalTok{(ggplot2)}
\end{Highlighting}
\end{Shaded}

\texttt{ggplot2} is defined at least by a \textbf{data set}, some
\textbf{aes,} which represent the ``aesthetic'' of the graphic, and the
type of the graphic thanks to the function \textbf{geom.}. The
\textbf{aes} is defined at least by a \textbf{x} and a \textbf{y}
values. It's possible to add also the \textbf{color}, the \textbf{fill},
the \textbf{group}. Some example will be showed below.

All the addition of function is add thanks the \textbf{+}

For more information: \url{https://bookdown.org/ael/rexplor/chap8.html}
(in French)

Graphic can caught in a variable but it is recommended to have one and
unique block per graphic:

\begin{Shaded}
\begin{Highlighting}[]
\CommentTok{\# A beautiful plot of this normal law CDF}
\FunctionTok{ggplot}\NormalTok{()}\SpecialCharTok{+} \CommentTok{\# start plot}
  \FunctionTok{xlim}\NormalTok{(xmin,xmax)}\SpecialCharTok{+} \CommentTok{\# define the range to plot}
  \FunctionTok{stat\_function}\NormalTok{(}\AttributeTok{fun=}\NormalTok{pnorm,}\AttributeTok{args=}\FunctionTok{c}\NormalTok{(}\AttributeTok{mean=}\NormalTok{my\_mu,}\AttributeTok{sd=}\NormalTok{my\_sg))}\SpecialCharTok{+} \CommentTok{\# define the normal law density}
  \FunctionTok{xlab}\NormalTok{(}\StringTok{"x"}\NormalTok{)}\SpecialCharTok{+} \CommentTok{\# decoration of the x axis}
  \FunctionTok{ylab}\NormalTok{(}\StringTok{"P(X\textless{}=x)"}\NormalTok{)}\SpecialCharTok{+} \CommentTok{\# decoration of the y axis}
  \FunctionTok{labs}\NormalTok{(}\AttributeTok{title=}\FunctionTok{paste}\NormalTok{(}\StringTok{"Normal law CDF {-} mean="}\NormalTok{,my\_mu,}\StringTok{", std dev="}\NormalTok{,}\FunctionTok{signif}\NormalTok{(my\_sg,}\DecValTok{3}\NormalTok{)))}\SpecialCharTok{+} \CommentTok{\# title of the plot}
  \FunctionTok{theme\_bw}\NormalTok{(}\AttributeTok{base\_size=}\DecValTok{12}\NormalTok{)}\SpecialCharTok{+} \CommentTok{\# ...the final touch}
  \FunctionTok{theme}\NormalTok{(}\AttributeTok{plot.title =} \FunctionTok{element\_text}\NormalTok{(}\AttributeTok{hjust =} \FloatTok{0.5}\NormalTok{, }\AttributeTok{color =} \StringTok{"blue"}\NormalTok{))}
\end{Highlighting}
\end{Shaded}

\includegraphics{R_Stats_Tuto_files/figure-latex/unnamed-chunk-27-1.pdf}

Here \texttt{stat\_function(fun=pnorm,args=c(mean=my\_mu,sd=my\_sg)} is
used to draw a function (\emph{the CDF of a normal law in this case}).

\hypertarget{the-point_plot}{%
\subsection{The point\_plot:}\label{the-point_plot}}

\begin{Shaded}
\begin{Highlighting}[]
\FunctionTok{plot}\NormalTok{(Ca,  }
  \AttributeTok{type =} \StringTok{"p"}\NormalTok{, }\CommentTok{\# type de tracé: points ("p"), lignes ("l"), les deux ("b" ou "o"),}
  \AttributeTok{col =} \StringTok{"blue"}\NormalTok{, }\CommentTok{\# couleur, tapez \textasciigrave{}colours()\textasciigrave{} pour la liste complète}
  \AttributeTok{pch =} \DecValTok{4}\NormalTok{, }\CommentTok{\# type de symboles, un chiffre entre 0 et 25, tapez \textasciigrave{}?points\textasciigrave{}}
  \AttributeTok{cex =} \FloatTok{0.5}\NormalTok{, }\CommentTok{\# taille des symboles}
  \AttributeTok{lty =} \DecValTok{3}\NormalTok{, }\CommentTok{\# type de lignes, un chiffre entre 1 et 6}
  \AttributeTok{lwd =} \FloatTok{1.2}\NormalTok{, }\CommentTok{\# taille de lignes}
  \AttributeTok{ylab =} \StringTok{"Ca content"}\NormalTok{, }\CommentTok{\# titre pour l\textquotesingle{}axe des y}
  \AttributeTok{main=}\StringTok{"Exercie 3 of the Statistic lecture"}\NormalTok{)}
\FunctionTok{abline}\NormalTok{(}\AttributeTok{h=}\FunctionTok{mean}\NormalTok{(Ca)}\SpecialCharTok{+}\DecValTok{3}\SpecialCharTok{*}\FunctionTok{sd}\NormalTok{(Ca), }\AttributeTok{col=}\StringTok{"red"}\NormalTok{) }\CommentTok{\# Add a line on the graphic}
\FunctionTok{text}\NormalTok{(}\DecValTok{15}\NormalTok{,}\DecValTok{200}\NormalTok{, }\StringTok{"3 * Standard Deviation"}\NormalTok{, }\AttributeTok{col=}\StringTok{"red"}\NormalTok{) }\CommentTok{\# Add a text on the graphic}
\end{Highlighting}
\end{Shaded}

\includegraphics{R_Stats_Tuto_files/figure-latex/unnamed-chunk-29-1.pdf}

\texttt{abline} \& \texttt{text} functions can add a line and a text
respectively in the graphic.

With \texttt{ggplot2} use \texttt{geom.point()}:

\begin{Shaded}
\begin{Highlighting}[]
\CommentTok{\# Example of data frame}
\NormalTok{df\_binom}\OtherTok{\textless{}{-}}\FunctionTok{data.frame}\NormalTok{(}\AttributeTok{x=}\FunctionTok{c}\NormalTok{(}\DecValTok{0}\SpecialCharTok{:}\NormalTok{n),}\AttributeTok{p=}\FunctionTok{dbinom}\NormalTok{(}\AttributeTok{x=}\FunctionTok{c}\NormalTok{(}\DecValTok{0}\SpecialCharTok{:}\NormalTok{n),}\AttributeTok{size=}\NormalTok{n,}\AttributeTok{prob=}\NormalTok{p))}
\CommentTok{\# Genaration of a graph}
\FunctionTok{ggplot}\NormalTok{(}\AttributeTok{data=}\NormalTok{df\_binom)}\SpecialCharTok{+}
  \FunctionTok{aes}\NormalTok{(}\AttributeTok{x=}\NormalTok{x,}\AttributeTok{y=}\NormalTok{p)}\SpecialCharTok{+}
  \FunctionTok{geom\_point}\NormalTok{()}\SpecialCharTok{+}
  \FunctionTok{geom\_segment}\NormalTok{(}\FunctionTok{aes}\NormalTok{(}\AttributeTok{x=}\NormalTok{x,}\AttributeTok{y=}\DecValTok{0}\NormalTok{,}\AttributeTok{xend=}\NormalTok{x,}\AttributeTok{yend=}\NormalTok{p))}\SpecialCharTok{+}
  \FunctionTok{theme\_light}\NormalTok{()}\SpecialCharTok{+}
  \FunctionTok{xlab}\NormalTok{(}\StringTok{"Number of success, x"}\NormalTok{)}\SpecialCharTok{+}
  \FunctionTok{ylab}\NormalTok{(}\StringTok{"Probability, p"}\NormalTok{)}\SpecialCharTok{+}
  \FunctionTok{labs}\NormalTok{(}\AttributeTok{title=}\FunctionTok{paste}\NormalTok{(}\StringTok{"Density of the binomial law, size="}\NormalTok{,n,}\StringTok{", prob="}\NormalTok{,p))}
\end{Highlighting}
\end{Shaded}

\includegraphics{R_Stats_Tuto_files/figure-latex/unnamed-chunk-30-1.pdf}

\hypertarget{the-barplot}{%
\subsection{The Barplot}\label{the-barplot}}

Basic R code for to generate a barplot of a Poison law:

\begin{Shaded}
\begin{Highlighting}[]
\FunctionTok{barplot}\NormalTok{(}\FunctionTok{rpois}\NormalTok{(}\DecValTok{100}\NormalTok{, }\AttributeTok{lambda =}\NormalTok{ m))}
\end{Highlighting}
\end{Shaded}

\includegraphics{R_Stats_Tuto_files/figure-latex/unnamed-chunk-31-1.pdf}

With \texttt{ggplot2} use \texttt{geom.histogram()}:

\begin{Shaded}
\begin{Highlighting}[]
\FunctionTok{ggplot}\NormalTok{(iris, }\FunctionTok{aes}\NormalTok{(}\AttributeTok{x =}\NormalTok{ Species, }\AttributeTok{y =}\NormalTok{ Sepal.Length))}\SpecialCharTok{+}
  \FunctionTok{geom\_bar}\NormalTok{(}\AttributeTok{stat =} \StringTok{"identity"}\NormalTok{)}\SpecialCharTok{+}
  \FunctionTok{theme\_test}\NormalTok{()}
\end{Highlighting}
\end{Shaded}

\includegraphics{R_Stats_Tuto_files/figure-latex/unnamed-chunk-32-1.pdf}

It is possible to have multigroup in the barplot. It's just needed to
add the parameter \texttt{fill\ =\ group2} in the \texttt{aes()}. By
default the group are stacked. To have the barplot side by side, add
\texttt{position=position\_dodge()} in the \texttt{geom\_bar} parameter.

More complicate, the case of 2 continue variables y. It is not possible
at this step with \texttt{geom\_bar} function, which allow only one x
for one y. To solve this issue, it's needed to reorganized the data to
transform a table with y columns to a table with only two columns. This
transformation can be done with the function
\texttt{gather(data,\ key,\ value,\ columns\_to\_gather)} of the library
\texttt{tidyr}.

\begin{itemize}
\tightlist
\item
  \textbf{data} ==\textgreater{} data frame, example iris
\item
  \textbf{key} ==\textgreater{} name of the group
\item
  \textbf{value} ==\textgreater{} name of the y axis
\item
  \textbf{columns\_to\_gather} ==\textgreater{} y variables of the
  original data frame that it needed to be merged, separate by
  \texttt{,}\\
\end{itemize}

\begin{Shaded}
\begin{Highlighting}[]
\FunctionTok{library}\NormalTok{(tidyr)}
\NormalTok{reformat }\OtherTok{\textless{}{-}} \FunctionTok{gather}\NormalTok{(iris, }\AttributeTok{key =} \StringTok{"Part\_flower"}\NormalTok{, }\AttributeTok{value =} \StringTok{"Length"}\NormalTok{, Sepal.Length, Petal.Length)}
\FunctionTok{head}\NormalTok{(reformat)}
\end{Highlighting}
\end{Shaded}

\begin{verbatim}
##   Sepal.Width Petal.Width Species  Part_flower Length
## 1         3.5         0.2  setosa Sepal.Length    5.1
## 2         3.0         0.2  setosa Sepal.Length    4.9
## 3         3.2         0.2  setosa Sepal.Length    4.7
## 4         3.1         0.2  setosa Sepal.Length    4.6
## 5         3.6         0.2  setosa Sepal.Length    5.0
## 6         3.9         0.4  setosa Sepal.Length    5.4
\end{verbatim}

\begin{Shaded}
\begin{Highlighting}[]
\FunctionTok{ggplot}\NormalTok{(reformat, }\FunctionTok{aes}\NormalTok{(}\AttributeTok{x =}\NormalTok{ Species, }\AttributeTok{y =}\NormalTok{ Length, }\AttributeTok{fill=}\NormalTok{Part\_flower))}\SpecialCharTok{+}
  \FunctionTok{geom\_bar}\NormalTok{(}\AttributeTok{stat =} \StringTok{"identity"}\NormalTok{, }\AttributeTok{position=}\FunctionTok{position\_dodge}\NormalTok{())}\SpecialCharTok{+}
  \FunctionTok{theme\_test}\NormalTok{()}
\end{Highlighting}
\end{Shaded}

\includegraphics{R_Stats_Tuto_files/figure-latex/unnamed-chunk-33-1.pdf}

More inspiring barplot :
\url{https://www.datanovia.com/en/fr/lessons/ggplot-barplot/}

\emph{Remark: it is not the best graph to describe the iris database}

\hypertarget{the-histogram}{%
\subsection{The Histogram}\label{the-histogram}}

Basic R code for to generate an histogram of a nominal law:

\begin{Shaded}
\begin{Highlighting}[]
\FunctionTok{hist}\NormalTok{(}\FunctionTok{rnorm}\NormalTok{(}\DecValTok{100}\NormalTok{,}\AttributeTok{mean=}\NormalTok{my\_mu,}\AttributeTok{sd=}\NormalTok{my\_sg), }\AttributeTok{freq=}\NormalTok{F) }
\FunctionTok{curve}\NormalTok{(}\FunctionTok{dnorm}\NormalTok{(x,}\AttributeTok{mean=}\NormalTok{my\_mu,}\AttributeTok{sd=}\NormalTok{my\_sg),}\AttributeTok{from =}\NormalTok{ xmin,}\AttributeTok{to =}\NormalTok{ xmax,}\AttributeTok{ylab=}\StringTok{"densité"}\NormalTok{,}\AttributeTok{add=}\NormalTok{T,}\AttributeTok{col=}\StringTok{"red"}\NormalTok{)}
\end{Highlighting}
\end{Shaded}

\includegraphics{R_Stats_Tuto_files/figure-latex/unnamed-chunk-34-1.pdf}

With \texttt{ggplot2} use \texttt{geom.histogram()}:

\begin{Shaded}
\begin{Highlighting}[]
\NormalTok{data }\OtherTok{\textless{}{-}} \FunctionTok{data.frame}\NormalTok{(}\AttributeTok{x =} \FunctionTok{rnorm}\NormalTok{(}\DecValTok{100}\NormalTok{,}\AttributeTok{mean=}\NormalTok{my\_mu,}\AttributeTok{sd=}\NormalTok{my\_sg))}
\FunctionTok{ggplot}\NormalTok{(data, }\FunctionTok{aes}\NormalTok{(x))}\SpecialCharTok{+}
  \FunctionTok{geom\_histogram}\NormalTok{(}\FunctionTok{aes}\NormalTok{ (}\AttributeTok{y=}\FunctionTok{after\_stat}\NormalTok{(density)), }\CommentTok{\#after\_stat(density) normalize to 1}
                 \AttributeTok{fill =} \StringTok{"grey"}\NormalTok{,}
                 \AttributeTok{color =} \StringTok{"black"}\NormalTok{,}
                 \AttributeTok{bins =} \DecValTok{50}\NormalTok{)}\SpecialCharTok{+} \CommentTok{\# bins number of interval un the histogram}
  \FunctionTok{stat\_function}\NormalTok{(}\AttributeTok{fun=}\NormalTok{dnorm,}\AttributeTok{args=}\FunctionTok{c}\NormalTok{(}\AttributeTok{mean=}\NormalTok{my\_mu,}\AttributeTok{sd=}\NormalTok{my\_sg),}
                \AttributeTok{color =}\StringTok{"red"}\NormalTok{, }
                \AttributeTok{size =} \FloatTok{0.1}\NormalTok{)}\SpecialCharTok{+}
  \FunctionTok{xlim}\NormalTok{(xmin,xmax)}\SpecialCharTok{+}
  \FunctionTok{theme\_bw}\NormalTok{()}
\end{Highlighting}
\end{Shaded}

\begin{verbatim}
## Warning: Using `size` aesthetic for lines was deprecated in ggplot2 3.4.0.
## i Please use `linewidth` instead.
## This warning is displayed once every 8 hours.
## Call `lifecycle::last_lifecycle_warnings()` to see where this warning was
## generated.
\end{verbatim}

\begin{verbatim}
## Warning: Removed 2 rows containing missing values (`geom_bar()`).
\end{verbatim}

\includegraphics{R_Stats_Tuto_files/figure-latex/unnamed-chunk-35-1.pdf}

\hypertarget{the-box-plot}{%
\subsection{The box-plot}\label{the-box-plot}}

The boxplot can represent the distribution of the observation in
function of the group. It gives also the quantile, the mean and the
median information.

The easyiest way to have a boxplot is to use the basic R command:
\texttt{boxplot(quantitative\_variable\textasciitilde{}quatative\_variable,\ labels)}.
Between the quantitative\_variable \& the quantative\_variable there is
a (\textbf{\textasciitilde{}}).\\
Example:

\begin{Shaded}
\begin{Highlighting}[]
\FunctionTok{boxplot}\NormalTok{(iris}\SpecialCharTok{$}\NormalTok{Sepal.Length}\SpecialCharTok{\textasciitilde{}}\NormalTok{iris}\SpecialCharTok{$}\NormalTok{Species,}\AttributeTok{ylab=}\StringTok{"Sepal Length"}\NormalTok{,}\AttributeTok{xlab=}\StringTok{"Species"}\NormalTok{)}
\end{Highlighting}
\end{Shaded}

With \texttt{ggplot2} use \texttt{geom.boxplot()}:

\begin{Shaded}
\begin{Highlighting}[]
\NormalTok{p }\OtherTok{\textless{}{-}} \FunctionTok{ggplot}\NormalTok{(iris, }\FunctionTok{aes}\NormalTok{(}\AttributeTok{x=}\NormalTok{Species, }\AttributeTok{y=}\NormalTok{Sepal.Length, }\AttributeTok{fill=}\NormalTok{Species))}\SpecialCharTok{+}
  \FunctionTok{ylim}\NormalTok{(}\DecValTok{4}\NormalTok{,}\DecValTok{8}\NormalTok{)}\SpecialCharTok{+}
  \FunctionTok{ylab}\NormalTok{(}\StringTok{"Sepal Length"}\NormalTok{)}\SpecialCharTok{+} 
  \FunctionTok{geom\_boxplot}\NormalTok{(}\AttributeTok{alpha=}\FloatTok{0.7}\NormalTok{, }\CommentTok{\#alpha for the transparence,}
               \AttributeTok{outlier.colour=}\StringTok{"red"}\NormalTok{,}
               \AttributeTok{outlier.shape=}\DecValTok{8}\NormalTok{,}\AttributeTok{outlier.size=}\DecValTok{4}\NormalTok{) }\SpecialCharTok{+}
  \FunctionTok{stat\_summary}\NormalTok{(}\AttributeTok{fun =}\NormalTok{ mean, }\AttributeTok{geom=}\StringTok{"point"}\NormalTok{, }\AttributeTok{shape=}\DecValTok{20}\NormalTok{, }\AttributeTok{size=}\DecValTok{8}\NormalTok{, }\AttributeTok{color=}\StringTok{"Yellow"}\NormalTok{) }\SpecialCharTok{+}
  \FunctionTok{geom\_point}\NormalTok{(}\AttributeTok{position=}\StringTok{"jitter"}\NormalTok{, }\AttributeTok{color=}\StringTok{"blue"}\NormalTok{, }\AttributeTok{alpha=}\NormalTok{.}\DecValTok{5}\NormalTok{)}\SpecialCharTok{+}
  \FunctionTok{scale\_fill\_brewer}\NormalTok{(}\AttributeTok{palette=}\StringTok{"Set2"}\NormalTok{)}\SpecialCharTok{+}
  \FunctionTok{theme\_classic}\NormalTok{()}
\NormalTok{p}
\end{Highlighting}
\end{Shaded}

\includegraphics{R_Stats_Tuto_files/figure-latex/unnamed-chunk-37-1.pdf}

The function \texttt{stat\_summary(fun.y\ =\ mean)} display the mean
value.\\

\hypertarget{multiple-grahs}{%
\subsection{Multiple grahs}\label{multiple-grahs}}

As it is showed, it is possible to combine several graphic in one by
addition of \texttt{goem} functions. But sometimes, it is needed to
separate the graphics. It is called the faceting. Two function can be
used:

\begin{itemize}
\tightlist
\item
  \texttt{facet\_warp(qualitative\_variable,\ nrow\ =\ number\ of\ ligne)}
\item
  \texttt{facet\_grid(ligne\ \textasciitilde{}column)}
\end{itemize}

\begin{Shaded}
\begin{Highlighting}[]
\FunctionTok{ggplot}\NormalTok{(iris, }\FunctionTok{aes}\NormalTok{(}\AttributeTok{x=}\NormalTok{Sepal.Width, }\AttributeTok{y=}\NormalTok{Sepal.Length, }\AttributeTok{shape =}\NormalTok{ Species, }\AttributeTok{color=}\NormalTok{Species))}\SpecialCharTok{+}
  \FunctionTok{theme\_bw}\NormalTok{()}\SpecialCharTok{+}
  \FunctionTok{geom\_smooth}\NormalTok{(}\AttributeTok{method=}\StringTok{"lm"}\NormalTok{, }\AttributeTok{formula=}\NormalTok{ y}\SpecialCharTok{\textasciitilde{}}\FunctionTok{poly}\NormalTok{(x,}\DecValTok{3}\NormalTok{), }\AttributeTok{se=}\ConstantTok{TRUE}\NormalTok{, }\AttributeTok{size =} \FloatTok{0.5}\NormalTok{) }\SpecialCharTok{+}
  \FunctionTok{geom\_point}\NormalTok{() }\SpecialCharTok{+}
  \FunctionTok{facet\_wrap}\NormalTok{(iris}\SpecialCharTok{$}\NormalTok{Species, }\AttributeTok{nrow =} \DecValTok{2}\NormalTok{)}
\end{Highlighting}
\end{Shaded}

\includegraphics{R_Stats_Tuto_files/figure-latex/unnamed-chunk-38-1.pdf}

\begin{Shaded}
\begin{Highlighting}[]
\FunctionTok{ggplot}\NormalTok{(iris, }\FunctionTok{aes}\NormalTok{(}\AttributeTok{x=}\NormalTok{Sepal.Width, }\AttributeTok{y=}\NormalTok{Sepal.Length, }\AttributeTok{shape =}\NormalTok{ Species, }\AttributeTok{color =}\NormalTok{ Species))}\SpecialCharTok{+}
  \FunctionTok{theme\_bw}\NormalTok{()}\SpecialCharTok{+}
  \FunctionTok{geom\_point}\NormalTok{() }\SpecialCharTok{+} 
  \FunctionTok{facet\_grid}\NormalTok{(}\SpecialCharTok{\textasciitilde{}}\NormalTok{Species)}
\end{Highlighting}
\end{Shaded}

\includegraphics{R_Stats_Tuto_files/figure-latex/unnamed-chunk-39-1.pdf}

In the first facing graph, the function
\texttt{geom\_smooth(method="lm",\ formula=\ y\textbackslash{}\textasciitilde{}poly(x,3),\ se=TRUE,\ size\ =\ 0.5)}
has been introduced. This function is used to add a smooth regression
curve to show the general trend of the dataset. Here the method used to
draw this curve is a fitting linear model \textbf{lm.}

For more inspired graphs:\\
\url{https://rstudio-pubs-static.s3.amazonaws.com/578122_5e69256788bb4dcca6157d2bcfa7694e.html}\strut \\
\url{https://www.charlesbordet.com/fr/faire-beaux-graphiques-ggplot2/\#it\%C3\%A9ration-4---ajouter-des-couleurs-pour-chaque-groupe}\strut \\

\begin{center}\rule{0.5\linewidth}{0.5pt}\end{center}

\hypertarget{statistics-test}{%
\section{Statistics Test}\label{statistics-test}}

\hypertarget{test-de-student}{%
\subsection{\texorpdfstring{Test de
\textbf{Student}}{Test de Student}}\label{test-de-student}}

\hypertarget{test-de-fisher}{%
\subsection{\texorpdfstring{Test de
\textbf{Fisher}}{Test de Fisher}}\label{test-de-fisher}}

voir la library \texttt{moment} kurtosis skweness

\end{document}
